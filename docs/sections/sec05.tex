\chapter{Valutazioni Finali}

\section{Conclusioni}
La creazione di questa ontologia dedicata alle tecnologie Big Data e al processo di Data Science pensiamo possa essere uno strumento di supporto al processo decisionale efficace. L'ontologia fornisce un insieme ben strutturato di classi, proprietà e regole SWRL che consentono di estrarre informazioni rilevanti sulle tecnologie, i software e le operazioni nel dominio del Big Data.\\

Nonostante sia ancora incompleta, la sua progettazione offre una base robusta per future espansioni e sviluppi. L'obiettivo iniziale di incorporare un ampio spettro di informazioni in modo logico per rendere l'ontologia accessibile anche agli utenti esterni è stato raggiunto. Va sottolineato che, essendo un progetto universitario, l'attenzione principale è stata rivolta alla comprensione approfondita del Web Semantico, esplorando diverse componenti di ciascun aspetto, come le regole SWRL e le query SPARQL. In aggiunta, nel corso dell'approfondimento, abbiamo dovuto confrontarci con le limitazioni degli strumenti utilizzati, spingendoci a ricorrere a nuove tecniche o tecnologie. Questa sfida ci ha motivato ad ampliare ulteriormente le nostre conoscenze. Nel complesso, siamo soddisfatti del risultato ottenuto e riteniamo di aver raggiunto l'obiettivo prefissato.

\section{Sviluppi futuri}
Per migliorare ulteriormente l'efficacia dell'ontologia, sarebbe benefico ampliare il dominio con una gamma più ampia di tecnologie e linguaggi di programmazione, nonché incorporare gli ultimi sviluppi legati alla Data Science. Questo arricchimento consentirebbe di ottenere una panoramica più dettagliata delle tecnologie utilizzate nel contesto dei Big Data e agevolerebbe il processo di sviluppo di progetti di Data Science. Introducendo nuove definizioni di classi, insieme alla creazione di proprietà e regole supplementari, si potrebbe modellare scenari più complessi e specifici, migliorando la precisione delle inferenze effettuate dal reasoner.